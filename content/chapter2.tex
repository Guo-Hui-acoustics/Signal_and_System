\section{Lecture1}
    \subsection{LTI System Properties--Linearity}
    What is \textcolor{red}{LTI System}?

    It means \textbf{linear} and \textbf{time-invariant}. And this definition gives two properties of LTI.

    Suppose we have a LTI system $H$, 
    you should know that the nature of the system is independent of the input.
    So we can give $H$ an arbitrary input $x$, 
    correspondingly we have an output $y$.
        \inserttikzpicture
                {
                    \tikzset{
                        node distance=2.5cm,
                        block/.style={
                            rectangle, 
                            draw, 
                            thick, 
                            minimum height=1.5cm, 
                            minimum width=1.5cm, 
                            text centered
                        },
                        text_node/.style={
                            align=center
                        },
                        arrow/.style={
                            -Stealth, % 使用 arrows.meta 库定义的箭头样式
                            thick
                        }
                    }
                    % 2. 将原代码中的节点和绘图指令放在这里
                    % 定义各个节点 (Node)
                    \node[text_node] (input) {input signal $x$};
                    \node[block, right=of input] (system) {H};
                    \node[text_node, right=of system] (output) {output signal $y$};
                    \node[below=0.5cm of system] (system_label) {\textbf{LTI system}};
                    % 绘制箭头连接节点
                    \draw[arrow] (input) -- (system);
                    \draw[arrow] (system) -- (output);
                    \draw[arrow] (system.south) -- (system_label.north); % 连接锚点以获得更精确的箭头位置
                }
                {LTI system}
                {fig:LTI_system_model}

    First, \textbf{Linearity}. This property contains two sub-properties:
        \begin{itemize}
            \item \textbf{Scaling property}: 
            
            Suppose we have an arbitrary scalar $\alpha$. 
            Then if we expand input by $\alpha$ times, the output will also be expanded by $\alpha$ times.

            \inserttikzpicture
                {
                    \tikzset{
                        node distance=2.5cm,
                        block/.style={
                            rectangle, 
                            draw, 
                            thick, 
                            minimum height=1.5cm, 
                            minimum width=1.5cm, 
                            text centered
                        },
                        text_node/.style={
                            align=center
                        },
                        arrow/.style={
                            -Stealth, % 使用 arrows.meta 库定义的箭头样式
                            thick
                        }
                    }
                    % 2. 将原代码中的节点和绘图指令放在这里
                    % 定义各个节点 (Node)
                    \node[text_node] (input) {$\alpha x$};
                    \node[block, right=of input] (system) {H};
                    \node[text_node, right=of system] (output) {$\alpha y$};
                    \node[below=0.5cm of system] (system_label) {\textbf{LTI system}};
                    % 绘制箭头连接节点
                    \draw[arrow] (input) -- (system);
                    \draw[arrow] (system) -- (output);
                    \draw[arrow] (system.south) -- (system_label.north); % 连接锚点以获得更精确的箭头位置
                }
                {Scaling Property}
                {fig:scaling_property}

            \item \textbf{Additivity}
            
            Suppose we have the following input-output conbination, all the inputs are arbitrary:

            \inserttikzpicture
                {
                    \tikzset{
                        node distance=2.5cm,
                        block/.style={
                            rectangle, 
                            draw, 
                            thick, 
                            minimum height=1.5cm, 
                            minimum width=1.5cm, 
                            text centered
                        },
                        text_node/.style={
                            align=center
                        },
                        arrow/.style={
                            -Stealth, % 使用 arrows.meta 库定义的箭头样式
                            thick
                        }
                    }
                    % 2. 将原代码中的节点和绘图指令放在这里
                    % 定义各个节点 (Node)
                    \node[text_node] (input) {$x_1$};
                    \node[block, right=of input] (system) {H};
                    \node[text_node, right=of system] (output) {$y_1$};
                    \node[below=0.5cm of system] (system_label) {\textbf{LTI system}};
                    % 绘制箭头连接节点
                    \draw[arrow] (input) -- (system);
                    \draw[arrow] (system) -- (output);
                    \draw[arrow] (system.south) -- (system_label.north); % 连接锚点以获得更精确的箭头位置
                }
                {Combination1}
                {fig:combination1}

                \inserttikzpicture
                {
                    \tikzset{
                        node distance=2.5cm,
                        block/.style={
                            rectangle, 
                            draw, 
                            thick, 
                            minimum height=1.5cm, 
                            minimum width=1.5cm, 
                            text centered
                        },
                        text_node/.style={
                            align=center
                        },
                        arrow/.style={
                            -Stealth, % 使用 arrows.meta 库定义的箭头样式
                            thick
                        }
                    }
                    % 2. 将原代码中的节点和绘图指令放在这里
                    % 定义各个节点 (Node)
                    \node[text_node] (input) {$x_2$};
                    \node[block, right=of input] (system) {H};
                    \node[text_node, right=of system] (output) {$y_2$};
                    \node[below=0.5cm of system] (system_label) {\textbf{LTI system}};
                    % 绘制箭头连接节点
                    \draw[arrow] (input) -- (system);
                    \draw[arrow] (system) -- (output);
                    \draw[arrow] (system.south) -- (system_label.north); % 连接锚点以获得更精确的箭头位置
                }
                {Combination2}
                {fig:combination2}

                Then additivity is manifested as :
                \inserttikzpicture
                {
                    \tikzset{
                        node distance=2.5cm,
                        block/.style={
                            rectangle, 
                            draw, 
                            thick, 
                            minimum height=1.5cm, 
                            minimum width=1.5cm, 
                            text centered
                        },
                        text_node/.style={
                            align=center
                        },
                        arrow/.style={
                            -Stealth, % 使用 arrows.meta 库定义的箭头样式
                            thick
                        }
                    }
                    % 2. 将原代码中的节点和绘图指令放在这里
                    % 定义各个节点 (Node)
                    \node[text_node] (input) {$x_1 + x_2$};
                    \node[block, right=of input] (system) {H};
                    \node[text_node, right=of system] (output) {$y_1 + y_2$};
                    \node[below=0.5cm of system] (system_label) {\textbf{LTI system}};
                    % 绘制箭头连接节点
                    \draw[arrow] (input) -- (system);
                    \draw[arrow] (system) -- (output);
                    \draw[arrow] (system.south) -- (system_label.north); % 连接锚点以获得更精确的箭头位置
                }
                {Additivity}
                {fig:additivity}

        \end{itemize}
    
    Then, let's take a look at one example. There is a 2-point moving average filter (DT):
        \begin{equation}
            y[n] = \frac{x[n-1] + x[n]}{2}
        \end{equation}
    Does it have the linearity?

    \textbf{Proof.}

    Suppose there is an arbitrary input-output combination:
        \inserttikzpicture
                {
                    \tikzset{
                        node distance=2.5cm,
                        block/.style={
                            rectangle, 
                            draw, 
                            thick, 
                            minimum height=1.5cm, 
                            minimum width=1.5cm, 
                            text centered
                        },
                        text_node/.style={
                            align=center
                        },
                        arrow/.style={
                            -Stealth, % 使用 arrows.meta 库定义的箭头样式
                            thick
                        }
                    }
                    % 2. 将原代码中的节点和绘图指令放在这里
                    % 定义各个节点 (Node)
                    \node[text_node] (input) {$x[n]$};
                    \node[block, right=of input] (system) {H};
                    \node[text_node, right=of system] (output) {$y[n] = \frac{x[n-1]+x[n]}{2}$};
                    %\node[below=0.5cm of system] (system_label) {\textbf{LTI system}};
                    % 绘制箭头连接节点
                    \draw[arrow] (input) -- (system);
                    \draw[arrow] (system) -- (output);
                    %\draw[arrow] (system.south) -- (system_label.north); % 连接锚点以获得更精确的箭头位置
                }
                {Example1}
                {fig:example1}
    
    Now verify the scaling property, let:
        \begin{equation}
            \hat{x}[n] = \alpha x[n], \quad \alpha \in \mathbb{C}
        \end{equation}

    And the input-output relation is:
        \inserttikzpicture
                {
                    \tikzset{
                        node distance=2.5cm,
                        block/.style={
                            rectangle, 
                            draw, 
                            thick, 
                            minimum height=1.5cm, 
                            minimum width=1.5cm, 
                            text centered
                        },
                        text_node/.style={
                            align=center
                        },
                        arrow/.style={
                            -Stealth, % 使用 arrows.meta 库定义的箭头样式
                            thick
                        }
                    }
                    % 2. 将原代码中的节点和绘图指令放在这里
                    % 定义各个节点 (Node)
                    \node[text_node] (input) {$\hat{x}[n]$};
                    \node[block, right=of input] (system) {H};
                    \node[text_node, right=of system] (output) {$\hat{y}[n]$};
                    %\node[below=0.5cm of system] (system_label) {\textbf{LTI system}};
                    % 绘制箭头连接节点
                    \draw[arrow] (input) -- (system);
                    \draw[arrow] (system) -- (output);
                    %\draw[arrow] (system.south) -- (system_label.north); % 连接锚点以获得更精确的箭头位置
                }
                {Hat Relation}
                {fig:hat_relation}

    From the definitio of the system, we know:
                    \begin{equation}
                        \hat{y}[n] = \frac{\hat{x}[n-1] + \hat{x}[n]}{2}
                    \end{equation}
    
    As we have the following relation:
        \begin{equation}
            \hat{x}[n] = \alpha x[n], \quad \alpha \in \mathbb{C}
        \end{equation}

    So:
        \begin{equation}
            \begin{aligned}
            \hat{y}[n] =  &\frac{\alpha \hat{x}[n-1] +  \alpha \hat{x}[n] }{2} \\
                       =  &\alpha \frac{\hat{x}[n-1] + \hat{x}[n]}{2}\\
                       =  & \alpha y[n]
             \end{aligned}
        \end{equation}
    
    \textbf{End.}

    
    

    

    
    
    