\section{Lecture2}
    \subsection{System Properties--Linearity}
    What is {system}?

    The nature of the system is independent of the input, and it associates the input with the output. For example:
        \begin{equation}
            y[n] = x[n] + x[n-1]
        \end{equation}
    What is \textcolor{red}{Linearity}? This property contains two sub-properties:
        \begin{itemize}
            \item \textbf{Scaling property}: 
            
            Suppose we have an arbitrary scalar $\alpha$. 
            Then if we expand input by $\alpha$ times, the output will also be expanded by $\alpha$ times.

            \inserttikzpicture
                {
                    \tikzset{
                        node distance=2.5cm,
                        block/.style={
                            rectangle, 
                            draw, 
                            thick, 
                            minimum height=1.5cm, 
                            minimum width=1.5cm, 
                            text centered
                        },
                        text_node/.style={
                            align=center
                        },
                        arrow/.style={
                            -Stealth, % 使用 arrows.meta 库定义的箭头样式
                            thick
                        }
                    }
                    % 2. 将原代码中的节点和绘图指令放在这里
                    % 定义各个节点 (Node)
                    \node[text_node] (input) {$\alpha x$};
                    \node[block, right=of input] (system) {H};
                    \node[text_node, right=of system] (output) {$\alpha y$};
                    \node[below=0.5cm of system] (system_label) {\textbf{Linear system}};
                    % 绘制箭头连接节点
                    \draw[arrow] (input) -- (system);
                    \draw[arrow] (system) -- (output);
                    \draw[arrow] (system.south) -- (system_label.north); % 连接锚点以获得更精确的箭头位置
                }
                {Scaling Property}
                {fig:scaling_property}

            \item \textbf{Additivity}
            
            Suppose we have the following input-output conbination, all the inputs are arbitrary:

            \inserttikzpicture
                {
                    \tikzset{
                        node distance=2.5cm,
                        block/.style={
                            rectangle, 
                            draw, 
                            thick, 
                            minimum height=1.5cm, 
                            minimum width=1.5cm, 
                            text centered
                        },
                        text_node/.style={
                            align=center
                        },
                        arrow/.style={
                            -Stealth, % 使用 arrows.meta 库定义的箭头样式
                            thick
                        }
                    }
                    % 2. 将原代码中的节点和绘图指令放在这里
                    % 定义各个节点 (Node)
                    \node[text_node] (input) {$x_1$};
                    \node[block, right=of input] (system) {H};
                    \node[text_node, right=of system] (output) {$y_1$};
                    \node[below=0.5cm of system] (system_label) {\textbf{Linear system}};
                    % 绘制箭头连接节点
                    \draw[arrow] (input) -- (system);
                    \draw[arrow] (system) -- (output);
                    \draw[arrow] (system.south) -- (system_label.north); % 连接锚点以获得更精确的箭头位置
                }
                {Combination1}
                {fig:combination1}

                \inserttikzpicture
                {
                    \tikzset{
                        node distance=2.5cm,
                        block/.style={
                            rectangle, 
                            draw, 
                            thick, 
                            minimum height=1.5cm, 
                            minimum width=1.5cm, 
                            text centered
                        },
                        text_node/.style={
                            align=center
                        },
                        arrow/.style={
                            -Stealth, % 使用 arrows.meta 库定义的箭头样式
                            thick
                        }
                    }
                    % 2. 将原代码中的节点和绘图指令放在这里
                    % 定义各个节点 (Node)
                    \node[text_node] (input) {$x_2$};
                    \node[block, right=of input] (system) {H};
                    \node[text_node, right=of system] (output) {$y_2$};
                    \node[below=0.5cm of system] (system_label) {\textbf{Linear system}};
                    % 绘制箭头连接节点
                    \draw[arrow] (input) -- (system);
                    \draw[arrow] (system) -- (output);
                    \draw[arrow] (system.south) -- (system_label.north); % 连接锚点以获得更精确的箭头位置
                }
                {Combination2}
                {fig:combination2}

                Then additivity is manifested as :
                \inserttikzpicture
                {
                    \tikzset{
                        node distance=2.5cm,
                        block/.style={
                            rectangle, 
                            draw, 
                            thick, 
                            minimum height=1.5cm, 
                            minimum width=1.5cm, 
                            text centered
                        },
                        text_node/.style={
                            align=center
                        },
                        arrow/.style={
                            -Stealth, % 使用 arrows.meta 库定义的箭头样式
                            thick
                        }
                    }
                    % 2. 将原代码中的节点和绘图指令放在这里
                    % 定义各个节点 (Node)
                    \node[text_node] (input) {$x_1 + x_2$};
                    \node[block, right=of input] (system) {H};
                    \node[text_node, right=of system] (output) {$y_1 + y_2$};
                    \node[below=0.5cm of system] (system_label) {\textbf{Linear system}};
                    % 绘制箭头连接节点
                    \draw[arrow] (input) -- (system);
                    \draw[arrow] (system) -- (output);
                    \draw[arrow] (system.south) -- (system_label.north); % 连接锚点以获得更精确的箭头位置
                }
                {Additivity}
                {fig:additivity}

        \end{itemize}

    \subsection{Linearity Example1-2 Point Moving Average Filter}
    There is a 2-point moving average filter (DT):
        \begin{equation}
            y[n] = \frac{x[n-1] + x[n]}{2}
        \end{equation}
    Does it have the linearity?

    \noindent \textbf{Proof.}

    Suppose there is an arbitrary input-output combination:
        \inserttikzpicture
                {
                    \tikzset{
                        node distance=2.5cm,
                        block/.style={
                            rectangle, 
                            draw, 
                            thick, 
                            minimum height=1.5cm, 
                            minimum width=1.5cm, 
                            text centered
                        },
                        text_node/.style={
                            align=center
                        },
                        arrow/.style={
                            -Stealth, % 使用 arrows.meta 库定义的箭头样式
                            thick
                        }
                    }
                    % 2. 将原代码中的节点和绘图指令放在这里
                    % 定义各个节点 (Node)
                    \node[text_node] (input) {$x[n]$};
                    \node[block, right=of input] (system) {H};
                    \node[text_node, right=of system] (output) {$y[n] = \frac{x[n-1]+x[n]}{2}$};
                    %\node[below=0.5cm of system] (system_label) {\textbf{LTI system}};
                    % 绘制箭头连接节点
                    \draw[arrow] (input) -- (system);
                    \draw[arrow] (system) -- (output);
                    %\draw[arrow] (system.south) -- (system_label.north); % 连接锚点以获得更精确的箭头位置
                }
                {Example1}
                {fig:example1}
    
    Now verify the scaling property.
    
    Let $\hat{x}[n] = \alpha x[n], \quad \alpha \in \mathbb{C}$.

    And the input-output relation is:
        \inserttikzpicture
                {
                    \tikzset{
                        node distance=2.5cm,
                        block/.style={
                            rectangle, 
                            draw, 
                            thick, 
                            minimum height=1.5cm, 
                            minimum width=1.5cm, 
                            text centered
                        },
                        text_node/.style={
                            align=center
                        },
                        arrow/.style={
                            -Stealth, % 使用 arrows.meta 库定义的箭头样式
                            thick
                        }
                    }
                    % 2. 将原代码中的节点和绘图指令放在这里
                    % 定义各个节点 (Node)
                    \node[text_node] (input) {$\hat{x}[n]$};
                    \node[block, right=of input] (system) {H};
                    \node[text_node, right=of system] (output) {$\hat{y}[n]$};
                    %\node[below=0.5cm of system] (system_label) {\textbf{LTI system}};
                    % 绘制箭头连接节点
                    \draw[arrow] (input) -- (system);
                    \draw[arrow] (system) -- (output);
                    %\draw[arrow] (system.south) -- (system_label.north); % 连接锚点以获得更精确的箭头位置
                }
                {Hat Relation}
                {fig:hat_relation}

    From the definitio of the system, we know:
                    \begin{equation}
                        \hat{y}[n] = \frac{\hat{x}[n-1] + \hat{x}[n]}{2}
                    \end{equation}
    
    As we have the following relation:
        \begin{equation}
            \hat{x}[n] = \alpha x[n], \quad \alpha \in \mathbb{C}
        \end{equation}

    So:
        \begin{equation}
            \begin{aligned}
            \hat{y}[n] =  &\frac{\alpha \hat{x}[n-1] +  \alpha \hat{x}[n] }{2} \\
                       =  &\alpha \frac{\hat{x}[n-1] + \hat{x}[n]}{2}\\
                       =  & \alpha y[n]
             \end{aligned}
        \end{equation}

    Then let's verify the additivity.

    Let $\hat{x}[n] = x_1[n] + x_2[n]$.

    And the imput-output relation is:
        \inserttikzpicture
                {
                    \tikzset{
                        node distance=2.5cm,
                        block/.style={
                            rectangle, 
                            draw, 
                            thick, 
                            minimum height=1.5cm, 
                            minimum width=1.5cm, 
                            text centered
                        },
                        text_node/.style={
                            align=center
                        },
                        arrow/.style={
                            -Stealth, % 使用 arrows.meta 库定义的箭头样式
                            thick
                        }
                    }
                    % 2. 将原代码中的节点和绘图指令放在这里
                    % 定义各个节点 (Node)
                    \node[text_node] (input) {$x_1[n]$};
                    \node[block, right=of input] (system) {H};
                    \node[text_node, right=of system] (output) {$y_1[n]=\frac{x_1[n] + x_1[n-1]}{2}$};
                    %\node[below=0.5cm of system] (system_label) {\textbf{LTI system}};
                    % 绘制箭头连接节点
                    \draw[arrow] (input) -- (system);
                    \draw[arrow] (system) -- (output);
                    %\draw[arrow] (system.south) -- (system_label.north); % 连接锚点以获得更精确的箭头位置
                }
                {Relation1}
                {fig:relation1}

        \inserttikzpicture
                {
                    \tikzset{
                        node distance=2.5cm,
                        block/.style={
                            rectangle, 
                            draw, 
                            thick, 
                            minimum height=1.5cm, 
                            minimum width=1.5cm, 
                            text centered
                        },
                        text_node/.style={
                            align=center
                        },
                        arrow/.style={
                            -Stealth, % 使用 arrows.meta 库定义的箭头样式
                            thick
                        }
                    }
                    % 2. 将原代码中的节点和绘图指令放在这里
                    % 定义各个节点 (Node)
                    \node[text_node] (input) {$x_2[n]$};
                    \node[block, right=of input] (system) {H};
                    \node[text_node, right=of system] (output) {$y_2[n]=\frac{x_2[n] + x_2[n-1]}{2}$};
                    %\node[below=0.5cm of system] (system_label) {\textbf{LTI system}};
                    % 绘制箭头连接节点
                    \draw[arrow] (input) -- (system);
                    \draw[arrow] (system) -- (output);
                    %\draw[arrow] (system.south) -- (system_label.north); % 连接锚点以获得更精确的箭头位置
                }
                {Relation2}
                {fig:relation2}

        From the definitio of the system, we know:
                    \begin{equation}
                        \hat{y}[n] = \frac{\hat{x}[n-1] + \hat{x}[n]}{2}
                    \end{equation}

    As we have the following relation:
        \begin{equation}
            \hat{x}[n] = x_1[n] + x_2[n]
        \end{equation}
    
    So:
        \begin{equation}
            \begin{aligned}
            \hat{y}[n] =  &\frac{\alpha \hat{x}[n-1] +  \alpha \hat{x}[n] }{2} \\
                       =  &\frac{x_1[n] + x_2[n] + x_1[n-1] + x_2[n-1]}{2}\\
                       =  & y_1[n] + y_2[n]
             \end{aligned}
        \end{equation}
    
    So this system satisfies the linearity.

    \noindent \textbf{End.}

    \subsection{Linearity Example2-Median Filter}
    There is a median filter (DT):
        \begin{equation}
            y[n] = med\{x[n-1],x[n],x[n+1]\}
        \end{equation}
    Does it have the linearity?

    First, what is the median of a sequence of numbers? 
    Consider there are three numbers, and we want to get the median.

    If there are no equal numbers, the median is the number whose size is in the middle;
    If there are two or three equal numbers, the median is the average of equal numbers.

    Then, we should have an intuitive understanding of this filter. 
    Look at the following example:
        \inserttikzpicture
                    {
                        \begin{axis}[
                            % --- 坐标轴样式 ---
                            axis lines=middle,              % 坐标轴在原点 (0,0) 相交
                            xlabel={$n$},                   % 横轴名称
                            ylabel={$x[n]$},                % 纵轴名称
                            % --- 坐标轴范围与刻度 ---
                            % 扩大 xmin 和 xmax 为省略号留出空间
                            xmin=-3.5, xmax=2.5,
                            ymin=0, ymax=2.5,
                            xtick={-3, -2, -1, 0, 1, 2},
                            ytick=\empty,
                            % --- 标签位置样式 ---
                            xlabel style={at={(ticklabel* cs:1.0)}, anchor=north west},
                            ylabel style={at={(ticklabel* cs:1.0)}, anchor=east, rotate=0},
                            % --- 自定义节点(标签)的样式 ---
                            nodes near coords={\pgfplotspointmeta}, 
                            every node near coord/.style={
                                anchor=east, 
                                rotate=0,
                                black,
                                font=\small
                            }
                        ]
                        % --- 绘制火柴杆和标签 ---
                        \addplot+ [
                            ycomb, blue, mark=*, mark size=2pt, line width=1.2pt
                        ] 
                        table[
                            header=false,
                            x index=0,              % 使用第 0 列 (第一列) 作为 x
                            y index=1,              % 使用第 1 列 (第二列) 作为 y
                            meta index=2,             % 使用第 2 列 (第三列) 作为标签元数据
                            point meta=explicit symbolic,
                            row sep=crcr % 修正了行分隔符
                        ] {
                            % n     height   label
                           -3      0          {} \\
                           -2      2          {$2$} \\
                           -1      1          {$1$} \\
                            0      1.5        {$1.5$}  \\
                            1      2        {$2$}  \\
                            2      1.25        {$1.25$}  \\
                        };
                        \end{axis}
                }
                {Median Filter Input: $x[n]$}
                {fig:med_filter_input}

    \noindent How can we get $y[-2]$? 

    According to the definition, to get $y[-2]$, 
    we need $x[-3]-0$, $x[-2]-2$ and $x[-1]-1$. There are no equal numbers,
    so $y[-2]$ or the median is 1.

    Similarly, we obtain the outputs at the other time points:
        \inserttikzpicture
                    {
                        \begin{axis}[
                            % --- 坐标轴样式 ---
                            axis lines=middle,              % 坐标轴在原点 (0,0) 相交
                            xlabel={$n$},                   % 横轴名称
                            ylabel={$y[n]$},                % 纵轴名称
                            % --- 坐标轴范围与刻度 ---
                            % 扩大 xmin 和 xmax 为省略号留出空间
                            xmin=-3.5, xmax=2.5,
                            ymin=0, ymax=2.5,
                            xtick={-2, -1, 0, 1},
                            ytick=\empty,
                            % --- 标签位置样式 ---
                            xlabel style={at={(ticklabel* cs:1.0)}, anchor=north west},
                            ylabel style={at={(ticklabel* cs:1.0)}, anchor=east, rotate=0},
                            % --- 自定义节点(标签)的样式 ---
                            nodes near coords={\pgfplotspointmeta}, 
                            every node near coord/.style={
                                anchor=east, 
                                rotate=0,
                                black,
                                font=\small
                            }
                        ]
                        % --- 绘制火柴杆和标签 ---
                        \addplot+ [
                            ycomb, blue, mark=*, mark size=2pt, line width=1.2pt
                        ] 
                        table[
                            header=false,
                            x index=0,              % 使用第 0 列 (第一列) 作为 x
                            y index=1,              % 使用第 1 列 (第二列) 作为 y
                            meta index=2,             % 使用第 2 列 (第三列) 作为标签元数据
                            point meta=explicit symbolic,
                            row sep=crcr % 修正了行分隔符
                        ] {
                            % n     height   label
                           -2      1          {$1$} \\
                           -1      1.5          {$1.5$} \\
                            0      1.5        {$1.5$}  \\
                            1      1.5        {$1.5$}  \\
                        };
                        \end{axis}
                }
                {Median Filter Output: $y[n]$}
                {fig:med_filter_output}
    
    Now, give the proof.

    \noindent \textbf{Proof}.

    First, we should verify whether it satisfies the scaling property.

    Obviously, this property is satisfied. 
    Because enlarging the same group of numbers by the same multiple doesn't change 
    the order of size.

    Then, does it satisfy the additivity? Intuition tells us that it's not satisfied.

    If we want to prove that it's not satisfied, we only need to provide a counter example.

    This example can be an impulse and a shifted impulse:
        \inserttikzpicture
                {
                    \pgfplotsset{
                            discrete plot/.style={
                                ycomb,           % 绘制针状图
                                mark=*,          % 数据点标记为实心圆
                                mark size=2pt,   % 标记大小
                                line width=1pt,  % 线条宽度
                                axis lines=middle, % 坐标轴在中间交叉
                                xmin=-2.5, xmax=2.5, % x轴范围
                                ymin=-0.2, ymax=1.1, % y轴范围稍稍扩大,让'1'的刻度更清晰
                                xtick={-2,-1,0,1,2}, % x轴刻度
                                ytick={1},               % y轴刻度
                                enlarge x limits=0.1,    % x轴范围适当放大
                                enlarge y limits=0.2,    % y轴范围适当放大
                                title style={yshift=0.5em}, % 标题位置
                                xlabel={$n$},
                                xlabel style={
                                    at={(ticklabel* cs:1.0)},
                                    anchor=north west,
                                },
                                ylabel style={
                                    at={(ticklabel* cs:1.0)},
                                    anchor=west,
                                    rotate=0,
                                },
                            }
                        }
                        % ----- 上图: u[n] -----
                        \begin{axis}[
                            discrete plot,
                            ylabel={$x_1[n]$},
                            %title={单位阶跃信号 (Unit Step Signal)},
                            at={(0, 5.0cm)} % 调整了间距以适应标题
                        ]
                        % 使用蓝色绘制 u[n] 的数据点
                        \addplot+[blue] coordinates {
                            (-2, 0) (-1, 0) (0,0) (1,1) (2, 0) 
                        };
                        \end{axis}
                        % ----- 下图: u[n-1] -----
                        \begin{axis}[
                            discrete plot,
                            ylabel={$x_2[n]$},
                            %title={时移阶跃信号 (Time-Shifted Step)},
                            at={(0, 0cm)}
                        ]
                        % 使用绿色绘制 u[n-1] 的数据点
                        \addplot+[green] coordinates {
                            (-2, 0) (-1,0) (0,0) (1,1) (2, 0)
                        };
                        \end{axis}
                        \begin{axis}[
                            discrete plot,
                            ylabel={$\hat{x}[n] = x_1[n] + x_2[n]$},
                            %title={时移阶跃信号 (Time-Shifted Step)},
                            at={(0, -5cm)}
                        ]
                        % 使用绿色绘制 u[n-1] 的数据点
                        \addplot+[red] coordinates {
                            (-2, 0) (-1,0) (0,1) (1,1) (2, 0)
                        };
                        \end{axis}
                    }
                % 第二、三个大括号,分别是总标题和标签
                {a Counter Example}
                {fig:counter_example}
    Calculate their results after median filtering, you will find:
        \begin{equation}
            \hat{y}[n] \neq y_1[n] + y_2[n]
        \end{equation}

    \newpage
    \subsection{Time Invariance-TI}
    Look at the following input-output combination:
        \inserttikzpicture
                {
                    \tikzset{
                        node distance=2.5cm,
                        block/.style={
                            rectangle, 
                            draw, 
                            thick, 
                            minimum height=1.5cm, 
                            minimum width=1.5cm, 
                            text centered
                        },
                        text_node/.style={
                            align=center
                        },
                        arrow/.style={
                            -Stealth, % 使用 arrows.meta 库定义的箭头样式
                            thick
                        }
                    }
                    % 2. 将原代码中的节点和绘图指令放在这里
                    % 定义各个节点 (Node)
                    \node[text_node] (input) {$x[n]$};
                    \node[block, right=of input] (system) {H};
                    \node[text_node, right=of system] (output) {$y[n]$};
                    %\node[below=0.5cm of system] (system_label) {\textbf{LTI system}};
                    % 绘制箭头连接节点
                    \draw[arrow] (input) -- (system);
                    \draw[arrow] (system) -- (output);
                    %\draw[arrow] (system.south) -- (system_label.north); % 连接锚点以获得更精确的箭头位置
                }
                {Combination1}
                {fig:combination1}

        \inserttikzpicture
                {
                    \tikzset{
                        node distance=2.5cm,
                        block/.style={
                            rectangle, 
                            draw, 
                            thick, 
                            minimum height=1.5cm, 
                            minimum width=1.5cm, 
                            text centered
                        },
                        text_node/.style={
                            align=center
                        },
                        arrow/.style={
                            -Stealth, % 使用 arrows.meta 库定义的箭头样式
                            thick
                        }
                    }
                    % 2. 将原代码中的节点和绘图指令放在这里
                    % 定义各个节点 (Node)
                    \node[text_node] (input) {$\hat{x}[n]$};
                    \node[block, right=of input] (system) {H};
                    \node[text_node, right=of system] (output) {$\hat{y}[n]$};
                    %\node[below=0.5cm of system] (system_label) {\textbf{LTI system}};
                    % 绘制箭头连接节点
                    \draw[arrow] (input) -- (system);
                    \draw[arrow] (system) -- (output);
                    %\draw[arrow] (system.south) -- (system_label.north); % 连接锚点以获得更精确的箭头位置
                }
                {Combination2}
                {fig:combination2}
    
    For: 
        \begin{equation}
            \mathrm{every}\quad  x \in \mathbb{X}, \quad
            \mathrm{every}\quad  N \in \mathbb{Z}
        \end{equation}

    Let:
        \begin{equation}
            \hat{x}[n] = x[n-N] 
        \end{equation}

    If the system $H$ satisfies the time invariance, then:
        \begin{equation}
            \hat{y}[n] = y[n-N]
        \end{equation}
    
    Now can you give an example, satisfies the TI but doesn't satisfy the linearity? Yes, it's:
        \begin{equation}
            y[n] = x^2[n]
        \end{equation}
    
    \newpage
    \subsection{Impulse Response}
    If a system simulataneously satisfies the linearity and the TI, 
    then we call it \textcolor{red}{LTI System}.

    In the subsection \textbf{Why do We Need Impulses?}, 
    we have known that: Any signal can be represented as a linear combination of shifted impulses.
    
    In the system perspective, $\delta[n]$ is also very important, 
    and if we give the LTI system an impulse as the input, 
    the output is called the \textbf{Impulse Response}, 
    and the corresponding symbol is $h[n]$:
        \inserttikzpicture
                {
                    \tikzset{
                        node distance=2.5cm,
                        block/.style={
                            rectangle, 
                            draw, 
                            thick, 
                            minimum height=1.5cm, 
                            minimum width=1.5cm, 
                            text centered
                        },
                        text_node/.style={
                            align=center
                        },
                        arrow/.style={
                            -Stealth, % 使用 arrows.meta 库定义的箭头样式
                            thick
                        }
                    }
                    % 2. 将原代码中的节点和绘图指令放在这里
                    % 定义各个节点 (Node)
                    \node[text_node] (input) {$x[n]=\delta[n]$};
                    \node[block, right=of input] (system) {H};
                    \node[text_node, right=of system] (output) {$y[n]=h[n]$};
                    \node[below=0.5cm of system] (system_label) {\textbf{Linear system}};
                    % 绘制箭头连接节点
                    \draw[arrow] (input) -- (system);
                    \draw[arrow] (system) -- (output);
                    \draw[arrow] (system.south) -- (system_label.north); % 连接锚点以获得更精确的箭头位置
                }
                {LTI System}
                {fig:LTI_system}
    
    \noindent \textbf{FAB FACT}

    If I know the impulse response of an LTI system, 
    then I will know the system response of any arbitraty input.

    \noindent \textbf{Proof}.
    
    Suppose there is an LTI system $H$.

    Consider an arbitrary input-output combination:
        \begin{equation}
            \delta[n] \quad \xrightarrow{\mathrm{LTI \quad system}} \quad h[n]
        \end{equation}
    
    Give the impulse an arbitrary shift, according to the time-invariance:
        \begin{equation}
            \delta[n-k] \quad \xrightarrow{\mathrm{LTI \quad system}} \quad h[n-k], \quad k\in Z
        \end{equation}
    
    Give the shifted impulse an scaling scalar, according to the scaling property:
        \begin{equation}
            x[k] \delta[n-k] \quad \xrightarrow{\mathrm{LTI \quad system}} \quad x[k] h[n-k]
        \end{equation}
    
    According to the additivity, we have:
        \begin{equation}
            \sum_{k=-\infty}^{+\infty} x[k] \delta[n-k] \quad \xrightarrow{\mathrm{LTI \quad system}} \quad \sum_{k=-\infty}^{+\infty} x[k] h[n-k]
        \end{equation}

    Now you must have find that, the infinite series on the left can form an arbitrary signal,
    that is to say:
        \begin{equation}
            x[n] = \sum_{k=-\infty}^{+\infty} x[k]
        \end{equation}
    
    \noindent \textbf{End}.

    \noindent To give you an intutive understanding, look at the following example.

    There is an LTI system $H$, and we the impulse response is:
        \begin{equation}
            \delta[n] \quad \xrightarrow{\mathrm{LTI \quad system}} \quad h[n]
        \end{equation}
    And the following is the input:
        \inserttikzpicture
                    {
                        \begin{axis}[
                            % --- 坐标轴样式 ---
                            axis lines=middle,              % 坐标轴在原点 (0,0) 相交
                            xlabel={$n$},                   % 横轴名称
                            ylabel={$x[n]$},                % 纵轴名称
                            % --- 坐标轴范围与刻度 ---
                            % 扩大 xmin 和 xmax 为省略号留出空间
                            xmin=-3.5, xmax=2.5,
                            ymin=0, ymax=2.5,
                            xtick={-2, -1, 0, 1},
                            ytick=\empty,
                            % --- 标签位置样式 ---
                            xlabel style={at={(ticklabel* cs:1.0)}, anchor=north west},
                            ylabel style={at={(ticklabel* cs:1.0)}, anchor=east, rotate=0},
                            % --- 自定义节点(标签)的样式 ---
                            nodes near coords={\pgfplotspointmeta}, 
                            every node near coord/.style={
                                anchor=east, 
                                rotate=0,
                                black,
                                font=\small
                            }
                        ]
                        % --- 绘制火柴杆和标签 ---
                        \addplot+ [
                            ycomb, blue, mark=*, mark size=2pt, line width=1.2pt
                        ] 
                        table[
                            header=false,
                            x index=0,              % 使用第 0 列 (第一列) 作为 x
                            y index=1,              % 使用第 1 列 (第二列) 作为 y
                            meta index=2,             % 使用第 2 列 (第三列) 作为标签元数据
                            point meta=explicit symbolic,
                            row sep=crcr % 修正了行分隔符
                        ] {
                            % n     height   label
                           -2      1          {$1$} \\
                           -1      2          {$2$} \\
                            0      0.5        {$0.5$}  \\
                            1      1.5        {$1.5$}  \\
                        };
                        \end{axis}
                }
                {Median Filter Output: $y[n]$}
                {fig:med_filter_output}
    
    \noindent What is the output?

    \noindent \textbf{Solution}.

    Decompose the input:
        \begin{equation}
            x[n] = \delta[n+2] + 2\delta[n+1] + 0.5\delta[n] + 1.5\delta[n-1]                       
        \end{equation}
    An important point is: here every component is a signal, not a scalar.

    As the system has additivity and scaling property, we can input each component seperately,
    then get the corresponding output:
        \begin{equation}
            y[n] = (1)h[n+2] + (2)h[n+1] + (0.5)h[n] + (1.5)h[n-1]
        \end{equation}
    
    \noindent \textbf{End}.

    \subsection{Convolution}
    Considering an LTI system, if we have known the impulse response $h[n]$, then:
        \begin{equation}
            y[n] = \sum_{k=-\infty}^{+\infty} x[k] h[n-k]
        \end{equation}
    We give this operation a name-\textbf{Convollution}, and the corresponding symbols is $\ast$:
        \begin{equation}
            \begin{aligned}
                y[n] &= \sum_{k=-\infty}^{+\infty} x[k] h[n-k] \\
                     &= x \ast h
            \end{aligned}
        \end{equation}
    
    Now we can say: The output of an LTI system is the convolution of the input and the impulse response.

    \subsubsection{Is the convolution operation commutative?}

    \noindent \textbf{Proof}.

    Let $l=n-k$, and change the summation variable of the above equation from $k$ to $l$:
        \begin{equation}
            \begin{aligned}
                y[n] &= \sum_{l=-\infty}^{+\infty} x[n-l] h[l] \\
                     &= h \ast x
            \end{aligned}
        \end{equation}
    
    So the convolution is commutative.

    \noindent \textbf{End}.

    \subsubsection{Does the convolution operation have the associative law?}
    In other words, we want to prove:
        \begin{equation}
            (x \ast h_1) \ast h_2 = x \ast(h_1 \ast h_2) 
        \end{equation}

    \noindent \textbf{Proof}.

    Let:
        \begin{equation}
            \begin{aligned}
                y[n] &= (x\ast h_1)[n]\\
                     &= \sum_{k=-\infty}^{+\infty} x[k]h_1[n-k]
            \end{aligned}
        \end{equation}
    
    Then the left side of the equation (LSE) is:
        \begin{equation}
            \begin{aligned}
        \mathbf{LSE} &= (y \ast h_2)[n]\\
                     &= \sum_{m=-\infty}^{+\infty} \boxed{y[m]}h_2[n-m] \\
                     &= \sum_{m=-\infty}^{+\infty} \boxed{\sum_{k=-\infty}^{+\infty} x[k]h_1[m-k]} h_2[n-m] \\
                     &= \sum_{m=-\infty}^{+\infty} \sum_{k=-\infty}^{+\infty} x[k]h_1[m-k] h_2[n-m]
            \end{aligned}
        \end{equation}
    
    In the above-mentioned summation process, we calculate the $y[m]$ first, 
    so here $m$ is considered as a scalar and the summation variable is $k$, $h[n-m]$ is also a scalar:
        \begin{equation}
            \mathbf{LSE} =\sum_{m=-\infty}^{+\infty} \boxed{\sum_{k=-\infty}^{+\infty} (x[k]h_1[m-k] h_2[n-m])}
        \end{equation}
    
    We can also take out the box:
        \begin{equation}
            \mathbf{LSE} =\sum_{m=-\infty}^{+\infty} \sum_{k=-\infty}^{+\infty} x[k]h_1[m-k] h_2[n-m]
        \end{equation}
    
    Suppose the summation converges absolutely, we can swap the order of the summation:
        \begin{equation}
            \mathbf{LSE} =\sum_{k=-\infty}^{+\infty} \sum_{m=-\infty}^{+\infty}  x[k]h_1[m-k] h_2[n-m]
        \end{equation}
    
    Here we calculate the summation that uses $m$ as a variable, so $x[k]$ and $k$ is a scalar:
        \begin{equation}
            \mathbf{LSE} =\sum_{k=-\infty}^{+\infty}x[k] \boxed{\sum_{m=-\infty}^{+\infty}  h_1[m-k] h_2[n-m]}
        \end{equation}

    Let's process $\sum_{m=-\infty}^{+\infty}  h_1[m-k] h_2[n-m]$ first. Let $p=m-k$, so $m=p+k$:
        \begin{equation}
            \begin{aligned}
                \sum_{m=-\infty}^{+\infty}  h_1[m-k] h_2[n-m] &= \sum_{p=-\infty}^{+\infty}  h_1[p] h_2[n-(p+k)] \\
                                                              &= \sum_{p=-\infty}^{+\infty}  h_1[p] h_2[(n-k)-p] \\
                                                              &= (h_1 \ast h_2)[n-k]
            \end{aligned}
        \end{equation}
    
    So:
        \begin{equation}
            \begin{aligned}
                \mathbf{LSE} &=\sum_{k=-\infty}^{+\infty}x[k] \boxed{\sum_{m=-\infty}^{+\infty}  h_1[m-k] h_2[n-m]}\\
                             &= \sum_{k=-\infty}^{+\infty}x[k] (h_1 \ast h_2)[n-k]\\
                             &= x* (h_2\ast h_1)[n]
            \end{aligned}
        \end{equation}
    
    \noindent \textbf{End}.
    
    \subsubsection{How can we understand $\sum_{k=-\infty}^{+\infty} \sum_{m=-\infty}^{+\infty}  x[k]h_1[m-k] h_2[n-m]$? }

    Let:
        \begin{equation}
            T(m,k) = x[k]h_1[m-k] h_2[n-m]
        \end{equation}
    
    Look at the following picture:
        \inserttikzpicture
        {
                \tikzset{
                % 增加 x 和 y 的单位长度,为标签提供更多空间
                x=2cm, 
                y=2cm,
                % 结点样式:蓝色实心圆点
                dot/.style={circle, fill=blue, inner sep=1.5pt},
                % 标签样式
                lbl/.style={font=\strut}
            }
            
            % --- 1. 绘制网格线 ---
            % (我们绘制一个 3x3 的正方形区域,交点在 -1, 0, 1)
            
            % 绘制垂直线 (k-1, k, k+1)
            \foreach \x in {-1, 0, 1} {
                % 绘制线,并稍微延长
                \draw[gray] (\x, -1.5) -- (\x, 1.5);
            }
            % 绘制水平线 (m-1, m, m+1)
            \foreach \y in {-1, 0, 1} {
                % 绘制线,并稍微延长
                \draw[gray] (-1.5, \y) -- (1.5, \y);
            }

            % --- 2. 绘制所有9个结点 (交点) ---
            \foreach \x in {-1, 0, 1} {
                \foreach \y in {-1, 0, 1} {
                \node[dot] at (\x, \y) {};
                }
            }
            % --- 3. 放置对齐网格线的标签 ---
            % 放置列标签 (对齐 x = -1, 0, 1)
            % anchor=south 使标签位于坐标 (x, 1.5) 的正上方
            \node[lbl, anchor=south] at (-1, 1.5) {$k-1$};
            \node[lbl, anchor=south] at ( 0, 1.5) {$k$};
            \node[lbl, anchor=south] at ( 1, 1.5) {$k+1$};
            
            % 放置行标签 (对齐 y = 1, 0, -1)
            % anchor=east 使标签位于坐标 (-1.5, y) 的正左方
            \node[lbl, anchor=east] at (-1.5,  1) {$m-1$};
            \node[lbl, anchor=east] at (-1.5,  0) {$m$};
            \node[lbl, anchor=east] at (-1.5, -1) {$m+1$};

            % --- 4. 标注中心结点 (k列, m行) ---
            % 也就是 (x=0, y=0) 处
            % 我们在 (0,0) 结点的右侧放置标签,以免遮挡圆点
            \node[right=2mm, font=\strut] at (0, -0.2) {$T(m,k)$};
        }
        {Binary Summation}
        {fig:binary_sum}

    So the summation becomes:
        \begin{equation}
            \sum_{k=-\infty}^{+\infty} \sum_{m=-\infty}^{+\infty} T(m,k) = \sum_{k=-\infty}^{+\infty} \boxed{\sum_{m=-\infty}^{+\infty} T(m,k)}
        \end{equation}
    Actually the firsr summation sums all the elements at column $k$. After the summation finishes, we get a function about $k$,
    Then it sums using $k$.

    \subsubsection{A cascade of two LTI system.}

    Look at the following example:
        \inserttikzpicture
            {
                % --- TikZ 绘图代码开始 ---
                \tikzset{
                    % 定义节点间距
                    node distance=2.5cm, 
                    % 信号节点样式 (x[n], y[n], q[n])
                    signal_node/.style={
                        font=\normalsize
                    },
                    % LTI 系统方框样式 (F, G)
                    system_block/.style={
                        font=\normalsize,
                        rectangle,
                        draw, thick,
                        minimum width=1.5cm,
                        minimum height=1cm,
                        align=center
                    },
                    % 级联系统的外框样式
                    enclosing_box/.style={
                        rectangle,
                        draw,
                        dashed, % 使用虚线以区分
                        inner sep=0.5cm % 方框和内部元素的间距
                    },
                    % 箭头样式
                    arrow_style/.style={
                        -Stealth,
                        thick,
                        draw
                    }
                }

                % --- 1. 定义从左到右的4个主要节点 ---
                \node[signal_node] (input) {$x[n]$};
                \node[system_block, right=of input] (sys_F) {F};
                \node[system_block, right=of sys_F] (sys_G) {G};
                \node[signal_node, right=of sys_G] (output) {$y[n]$};

                % --- 2. 绘制箭头 ---
                \draw[arrow_style] (input) -- (sys_F);
                % 箭头 F -> G, 并在上方标注 q[n]
                \draw[arrow_style] (sys_F) -- 
                    node[above, midway, font=\normalsize] {$q[n]$} (sys_G);
                \draw[arrow_style] (sys_G) -- (output);
                % --- 3. 绘制包裹 F 和 G 的大方框 ---
                % 使用 fit 库, 创建一个刚好能"适应" F 和 G 的节点
                \node[enclosing_box, fit=(sys_F) (sys_G)] (cascade_box) {};
                % --- 4. 绘制从大方框向下的箭头, 并标注 h[n] ---
                % 从大方框的南锚点 (cascade_box.south) 向下画 1cm
                % 并在新位置的下方标注 h[n]
                \draw[arrow_style] (cascade_box.south) -- ++(0, -1cm) 
                    node[below] {$h[n]$};
                % --- TikZ 绘图代码结束 ---
            }
            {Cascade of two LTI systems, F and G}
            {fig:lti_cascade}
    
    According to the direction of the signal flow, we have:
        \begin{equation}
            y[n] = (x\ast f) \ast g
        \end{equation}
    
    As the convolution has the associative law, we have:
        \begin{equation}
            \begin{aligned}
                y[n] &= (x\ast f) \ast g \\
                     &= x \ast (f \ast g) 
            \end{aligned}
        \end{equation}
    So we have:
        \begin{equation}
            \begin{aligned}
            h &=  f \ast g \\
              &=  g \ast f
            \end{aligned}
        \end{equation}
    that is to say: we can change the order of $F$ and $G$.

    \subsection{Convolution Interpretation}
    In this interpretation, you should remember one sentence:

    \textbf{Instantaneous actions have continuous consequences.}

    What dose $\delta[n]$ mean?
        \inserttikzpicture
                        {% 使用 pgfplots 环境绘制单位冲激信号
                            \begin{axis}[
                                compat=1.17,
                                axis lines=middle,
                                ylabel={$\delta[n]$},       % y轴标签
                                ylabel style={              % y轴标签样式
                                    at={(ticklabel* cs:1.0)},
                                    anchor=east,
                                    rotate=0,
                                },
                                xlabel={$n$},               % x轴标签
                                xlabel style={              % x轴标签样式
                                    at={(ticklabel* cs:1.0)},
                                    anchor=north west,
                                },
                                % 设置合适的坐标轴范围
                                ymin=-0.5, ymax=1.5,
                                xmin=-4.5, xmax=4.5,
                                % 在x轴上显示整数刻度
                                xtick={-4, -3, -2, -1, 0, 1, 2, 3, 4},
                                ytick={1},                  % 仅在 y=1 处显示一个刻度以示高度
                                clip=false,
                            ]
                            % 使用 ycomb 样式来绘制“火柴杆”
                            % 单位冲激信号只在 n=0 处有值
                            \addplot+[
                                ycomb,
                                blue,
                                thick,
                                mark=*,
                                mark options={fill=blue},
                            ] coordinates {
                                (0, 1) % 关键数据点:n=0 时,值为 1
                            };
                            \end{axis}
                        }
                        {Delata: $\delta[n]$}
                        {fig:unit_impulse}
    It means, we give a very short impulse at time 0.

    Suppose $\delta[n]$ is given to an LTI system as an input, the impulse response is as follows:
        \inserttikzpicture
                {
                    % 使用 pgfplots 环境绘制 stem plot
                    \begin{axis}[
                        compat=1.17,
                        axis lines=middle,          % 坐标轴在原点交汇
                        ylabel={$h[n]$},            % y轴标签
                        ylabel style={              % y轴标签样式,与您之前的示例保持一致
                            at={(ticklabel* cs:1.0)},
                            anchor=west,
                            rotate=0,
                        },
                        xlabel=$n$,                 % x轴标签 (离散时间索引)
                        xlabel style={              % x轴标签样式
                            at={(ticklabel* cs:1.0)},
                            anchor=north west,      % 微调位置防止与坐标轴箭头重叠
                        },
                        % 为离散时间信号设置坐标轴范围
                        ymin=-2, ymax=3,
                        xmin=-0.5, xmax=5.5,
                        % 在x轴上显示整数刻度,这对于离散信号至关重要
                        xtick={ 0, 1, 2, 3, 4, 5},
                        ytick=\empty,               % 保持y轴的简洁,移除刻度
                        clip=false,                 % 允许绘图内容超出坐标轴范围
                    ]
                    % 使用 ycomb 样式来绘制“火柴杆”
                    % mark=* 表示在每个数据点顶端添加一个实心圆点
                    \addplot+[
                        ycomb,
                        blue, % 设置颜色
                        thick, % 加粗线条
                        mark=*,
                        mark options={fill=blue}, % 将标记点用蓝色填充
                    ] coordinates {
                        (0, 2)
                        (1, -1)
                        (2, -1.5)
                        (3, 1)
                        (4, 0)
                        (5, 0.5)
                    };
                    \end{axis}
                }
                {Impulse Response: $h[n]$}
                {fig:impulse_response}
    
    You can see, although we only give an impulse at time 0, 
    this impulse remains in effect for the rest of the time.
    
    If we give a shifted impulse $\delta[n-1]$ as an input, what will we get?
    Yes, according to the TI, we will get:
         \inserttikzpicture
                {
                    % 使用 pgfplots 环境绘制 stem plot
                    \begin{axis}[
                        compat=1.17,
                        axis lines=middle,          % 坐标轴在原点交汇
                        ylabel={$h[n-1]$},            % y轴标签
                        ylabel style={              % y轴标签样式,与您之前的示例保持一致
                            at={(ticklabel* cs:1.0)},
                            anchor=west,
                            rotate=0,
                        },
                        xlabel=$n$,                 % x轴标签 (离散时间索引)
                        xlabel style={              % x轴标签样式
                            at={(ticklabel* cs:1.0)},
                            anchor=north west,      % 微调位置防止与坐标轴箭头重叠
                        },
                        % 为离散时间信号设置坐标轴范围
                        ymin=-2, ymax=3,
                        xmin=-0.5, xmax=6.5,
                        % 在x轴上显示整数刻度,这对于离散信号至关重要
                        xtick={ 0, 1, 2, 3, 4, 5,6},
                        ytick=\empty,               % 保持y轴的简洁,移除刻度
                        clip=false,                 % 允许绘图内容超出坐标轴范围
                    ]
                    % 使用 ycomb 样式来绘制“火柴杆”
                    % mark=* 表示在每个数据点顶端添加一个实心圆点
                    \addplot+[
                        ycomb,
                        blue, % 设置颜色
                        thick, % 加粗线条
                        mark=*,
                        mark options={fill=blue}, % 将标记点用蓝色填充
                    ] coordinates {
                        (1, 2)
                        (2, -1)
                        (3, -1.5)
                        (4, 1)
                        (5, 0)
                        (6, 0.5)
                    };
                    \end{axis}
                }
                {Shifted Impulse Response: $h[n-1]$}
                {fig:shifted_impulse_response}           
        
    If we give an arbitrary input $x[n]$, 
    we know $x[n]$ is a linear combination of different shifted impulses:
                    \begin{equation}
                        x[n] = \sum_{k=-\infty}^{+\infty}x[k]\delta[n-k]
                    \end{equation}
    Here, you can understand $x[k]$ as a \textbf{weight} for different $\delta[n-k]$.
    Because when we calculate the sum, $x[k]$ is seen as a scalar.

    So each component in $x[n]$ or shifted impulses will have a lasting impact, 
    and we should sum all the impact from each $\delta[n-k]$.

    For example, now we want to get the output at 3 or $y[3]$, our input is:
                    \begin{equation}
                        x[n] = 3\delta[n] + 2\delta[n-1] + 3\delta[n-2]
                    \end{equation}
    we should sum all the components' impact at $y[3]$:
                    \begin{equation}
                        y[3] = 3h[3] + 2h[3-1] + 3h[3-2]
                    \end{equation}

    
        



    
    
    
    





    
    
    
    
    
    
    
    
    
    
    
    


    
    
    
        
    




    





    

    
    

    

    
    
    