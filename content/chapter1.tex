\section{Lecture1}
    \subsection{What is a Signal}
        In this subsection, we will introduce some notions in \textbf{SS} course.

        First, what is a signal? A signal is a \textbf{function}. Suppose \textbf{x} is a signal,
        then it can represent the following mapping relationship:
            \begin{equation}
                \begin{aligned}
                    \mathbb{R}(\mathrm{reals}) &\longrightarrow \mathbb{R}(\mathrm{reals}) \\
                    \mathbb{Z}(\mathrm{integers}) &\longrightarrow \mathbb{R}(\mathrm{reals}) \\
                    \mathbb{Z}(\mathrm{reals}) &\longrightarrow \mathbb{C}(\mathrm{complexes})
                \end{aligned}
            \end{equation}
        In the above formula,  
        we call the left \textbf{domain or input space}, we call the right \textbf{range or output sapce},
        referring to the definition in the function,

        At the same time, an element in domain is called \textbf{independent variable}, an element in range is called \textbf{dependent variable or the value of function}.
        
        After we give the doamin and the range, we need a \textbf{rule} to map every element in the domain to the range,
        which says what we operate on the the independent variable. The rule is also called the \textbf{function relation}.
        
        For example, considering the following euqation:
            \begin{equation}
                x(n) = \cos \frac{\pi}{4} n
            \end{equation}
        we have the following relations:
            \begin{itemize}
                \item \textbf{domain}: $\mathbb{Z}$, a set of all integers;
                \item \textbf{independent variable}: $n$, integers;
                \item \textbf{range}: all real numbers in [-1,1];
                \item \textbf{value of function}: $x(n)$;
                \item \textbf{function relation}: \textbf{\emph{x}}, $  \cos \frac{\pi}{4}[\cdot] $
            \end{itemize}
    
    \newpage
    \subsection{DT Signal and CT Signal}
        Now let's introduce \textbf{DT} and \textbf{CT}.

        For a signal \textbf{x}, if the domain is $\mathbb{Z}$ or the set of all integers, 
        then the signal \textbf{x} is called \textbf{Discrete-Time} signal (\textbf{DT}).
        For example:
            \begin{equation}
                x(n) = \cos \frac{\pi}{4} n,\quad n \in \mathbb{Z}
            \end{equation}
        
        For a signal \textbf{\emph{x}}, if the domain is $\mathbb{R}$ or the set of all reals, 
        then the signal \textbf{\emph{x}} is called \textbf{Continuous-Time} signal (\textbf{CT}).
        Also, you can call \textbf{CT signal} as \textbf{Analog signal}.
        For example:
            \begin{equation}
                x(t) = \mathrm{e}^{-t},\quad t \in \mathbb{R}
            \end{equation}

        Meanwhile:
            \begin{itemize}
                \item if we use $n$ as the independent variable, that means the signal is \textbf{DT};
                \item if we use $t$ as the independent variable, that means the signal is \textbf{CT};
            \end{itemize}
        these are the conventions.

        \textbf{Supplement}: Does the signal have to be the function of \textbf{only time}? The answer is \textbf{No}.
        For example, the independent variables can be \textbf{space coordinates}.
    
    \subsection{Discrete-Time Signal and Digital Signal}
        Sometimes, we don't distinguish \textbf{digital signal} and \textbf{discrete-time signal}, 
        but they still have obvious differences.

        In terms of digital signal, it still meets the definition of function. 
        However, everything (including independent variable in domain and dependent variable in range) are represented by digits.

        In terms of discrete-time signal, the independent variable in domain is discrete, 
        but the values of the function can be reals (such as infinite recurring decimal).

        But for the computers, we can't have infinite precision, so we need to quantify the numbers, 
        which will cause us to lose accurcy.

    \newpage
    \subsection{Examples and Conventions}
        Let's give some examples about \textbf{DT} and \textbf{CT} signal.

        First, there is a continuous time example, a speech signal:
            \inserttikzpicture
                { % 第一个参数: TikZ 绘图代码
                    \begin{axis}[
                        compat=1.17,
                        axis lines=middle,
                        xtick=\empty,
                        ytick=\empty,
                        ymin=-2,
                        ymax=2.2, 
                        xmin=0,
                        xmax=10.5,
                        clip=false, 
                        xlabel=$t$,
                        xlabel style={
                            at={(ticklabel* cs:1.0)},
                            anchor=north,
                        },
                        ylabel={$x(t)$},
                        ylabel style={
                            at={(ticklabel* cs:1.0)},
                            anchor=east,
                            rotate=0,
                        },
                    ]
                        \addplot[
                            domain=0:10,
                            samples=200,
                            smooth,
                            thick,
                            blue,
                        ] {sin(deg(2*x)) * (1 + 0.5*sin(deg(5*x)))};
                    \end{axis}
                }
                {A Speech Signal}
                {fig:speech_signal}
        \noindent and there is an exp-deacy signal:
            \inserttikzpicture
                {
                    % 使用您提供的坐标轴样式和布局
                    \begin{axis}[
                        compat=1.17,
                        axis lines=middle,
                        xtick=\empty,  % 移除 x 轴刻度
                        ytick=\empty,  % 移除 y 轴刻度
                        % 调整坐标轴范围以适应 e^{-t} 并在 t<0 时有足够空间
                        ymin=-1,
                        ymax=8, 
                        xmin=-2.5,
                        xmax=5,
                        clip=false, 
                        xlabel=$t$,
                        xlabel style={
                            at={(ticklabel* cs:1.0)},
                            anchor=north,
                        },
                        ylabel={$x(t)$},
                        ylabel style={
                            at={(ticklabel* cs:1.0)},
                            anchor=east,
                            rotate=0,
                        },
                    ]
                    % 绘制函数 x(t) = e^{-t},并扩展定义域至负数
                    \addplot[
                        domain=-2:4.5, % 设置绘图的定义域,包含负数
                        samples=200,
                        smooth,
                        thick,
                        blue,
                    ] {exp(-x)};
                    \end{axis}
                } 
                {$x(t)=e^{-t}$}
                {fig:exp_decay}
        \noindent For the \textbf{CT} signal, we have the following conventions:
            \begin{itemize}
                \item $x$ at time $t$: $x(t)$, refers to a specific value of the signal at time t;
                \item $x$ without any arguments: \textbf{\emph{x}}, refers to the entire signal in $(-\infty, \infty)$;
            \end{itemize}
        However, in practice, when we define a signal, $x(t)$ also refers to the entire signal, 
        not a specific value at time $t$.

        Second, there is a discrete time signal:
            \inserttikzpicture
                {
                    % 使用 pgfplots 环境绘制 stem plot
                    \begin{axis}[
                        compat=1.17,
                        axis lines=middle,          % 坐标轴在原点交汇
                        ylabel={$x(n)$},            % y轴标签
                        ylabel style={              % y轴标签样式,与您之前的示例保持一致
                            at={(ticklabel* cs:1.0)},
                            anchor=east,
                            rotate=0,
                        },
                        xlabel=$n$,                 % x轴标签 (离散时间索引)
                        xlabel style={              % x轴标签样式
                            at={(ticklabel* cs:1.0)},
                            anchor=north west,      % 微调位置防止与坐标轴箭头重叠
                        },
                        % 为离散时间信号设置坐标轴范围
                        ymin=-2, ymax=3,
                        xmin=-3.5, xmax=6.5,
                        % 在x轴上显示整数刻度,这对于离散信号至关重要
                        xtick={-3, -2, -1, 0, 1, 2, 3, 4, 5, 6},
                        ytick=\empty,               % 保持y轴的简洁,移除刻度
                        clip=false,                 % 允许绘图内容超出坐标轴范围
                    ]
                    % 使用 ycomb 样式来绘制“火柴杆”
                    % mark=* 表示在每个数据点顶端添加一个实心圆点
                    \addplot+[
                        ycomb,
                        blue, % 设置颜色
                        thick, % 加粗线条
                        mark=*,
                        mark options={fill=blue}, % 将标记点用蓝色填充
                    ] coordinates {
                        (-2, 1)
                        (-1, 2)
                        (0, 1.5)
                        (1, -1)
                        (2, -1.5)
                        (3, 1)
                        (4, 0)
                        (5, 0.5)
                    };
                    \end{axis}
                }
                {An Example for \textbf{DT} signal}
                {fig:discrete_signal_example}
        \noindent The picture above is called \textbf{lollipops} or \textbf{stem plots}, 
        which is designed specifically for the \textbf{DT} signal. 
        We can find that, it's discrete on the domain.
    
    \subsection{Some Important DT signal}
        Now we introduce some important DT signal.
                    
        First, \textcolor{red}{Kronecker Delta / DT Impulse}. 
        
        In Lollipop language, it is represented as delta of $n$:
                    \inserttikzpicture
                        {% 使用 pgfplots 环境绘制单位冲激信号
                            \begin{axis}[
                                compat=1.17,
                                axis lines=middle,
                                ylabel={$\delta[n]$},       % y轴标签
                                ylabel style={              % y轴标签样式
                                    at={(ticklabel* cs:1.0)},
                                    anchor=east,
                                    rotate=0,
                                },
                                xlabel={$n$},               % x轴标签
                                xlabel style={              % x轴标签样式
                                    at={(ticklabel* cs:1.0)},
                                    anchor=north west,
                                },
                                % 设置合适的坐标轴范围
                                ymin=-0.5, ymax=1.5,
                                xmin=-4.5, xmax=4.5,
                                % 在x轴上显示整数刻度
                                xtick={-4, -3, -2, -1, 0, 1, 2, 3, 4},
                                ytick={1},                  % 仅在 y=1 处显示一个刻度以示高度
                                clip=false,
                            ]
                            % 使用 ycomb 样式来绘制“火柴杆”
                            % 单位冲激信号只在 n=0 处有值
                            \addplot+[
                                ycomb,
                                blue,
                                thick,
                                mark=*,
                                mark options={fill=blue},
                            ] coordinates {
                                (0, 1) % 关键数据点:n=0 时,值为 1
                            };
                            \end{axis}
                        }
                        {Kronecker Delata: $\delta[n]$}
                        {fig:unit_impulse_signal}
        In mathmatical language, we can define $\delta[n]$ :
                \begin{equation}
                    \delta[n]=
                    \left\{
                        \begin{aligned}
                            1, \quad & n =0 \\
                            0, \quad & n \neq 0
                        \end{aligned}
                    \right.
                \end{equation}

        From this example, you can find that, we use \textcolor{red}{Square Brakets} to 
        enclose the variable $n$. 
        And $\delta[n]$ is the most fundamental unit in DT signal.

        Second, \textcolor{red}{DT Unit Step}.

        In Lollipop language, the picture is as follows:
                \inserttikzpicture
                        {
                            % 使用 pgfplots 环境绘制单位阶跃信号
                            \begin{axis}[
                                compat=1.17,
                                axis lines=middle,
                                ylabel={$u[n]$},            % y轴标签
                                ylabel style={              % y轴标签样式
                                    at={(ticklabel* cs:1.0)},
                                    anchor=east,
                                    rotate=0,
                                },
                                xlabel={$n$},                 % x轴标签
                                xlabel style={              % x轴标签样式
                                    at={(ticklabel* cs:1.0)},
                                    anchor=north west,
                                },
                                % 设置合适的坐标轴范围
                                ymin=-0.5, ymax=1.5,
                                xmin=-4.5, xmax=8.5, % 稍微增加 xmax 为省略号留出空间
                                % 在x轴上显示整数刻度
                                xtick={-4, -3, -2, -1, 0, 1, 2, 3, 4, 5, 6},
                                ytick={1},                  % 仅在 y=1 处显示一个刻度
                                clip=false,
                            ]
                            % 使用 ycomb 样式来绘制“火柴杆”
                            \addplot+[
                                ycomb,
                                blue,
                                thick,
                                mark=*,
                                mark options={fill=blue},
                            ] coordinates {
                                % u[n] 在 n>=0 时值为 1
                                (0, 1)
                                (1, 1)
                                (2, 1)
                                (3, 1)
                                (4, 1)
                                (5, 1) % 最后一个实际绘制的火柴杆
                            };
                            
                            % 添加省略号来表示信号向右无限延伸
                            \node[text width=1.5cm, align=left, font=\normalsize, anchor=west] at (axis cs: 5.8, 1) {$\dots \dots$};
                            \end{axis}
                        }
                        {DT Unit Step: $u[n]$}
                        {fig:unit_step_signal}
        In mathmatical language, we can define $u[n]$:
                \begin{equation}
                    u[n]=
                    \left\{
                        \begin{aligned}
                            1, \quad & n \geqslant 0 \\
                            0, \quad & n \leq 0
                        \end{aligned}
                    \right.
                \end{equation}

        Also, we can define CT unit step $u(t)$:
                \begin{equation}
                    u(t)=
                    \left\{
                        \begin{aligned}
                            1, \quad & t \geqslant 0 \\
                            0, \quad & t \leq 0
                        \end{aligned}
                    \right.
                \end{equation}
        \noindent and here is the picture:
                \inserttikzpicture
                    {
                        % 使用 pgfplots 环境绘制连续单位阶跃信号
                        \begin{axis}[
                            compat=1.17,
                            axis lines=middle,
                            ylabel={$u(t)$},            % y轴标签
                            ylabel style={              % y轴标签样式
                                at={(ticklabel* cs:1.0)},
                                anchor=east,
                                rotate=0,
                            },
                            xlabel={$t$},                 % x轴标签
                            xlabel style={              % x轴标签样式
                                at={(ticklabel* cs:1.0)},
                                anchor=north west,
                            },
                            % 设置合适的坐标轴范围
                            ymin=-0.5, ymax=1.5,
                            xmin=-4.5, xmax=5.5,
                            % 在x轴上显示整数刻度
                            xtick={-4, -3, -2, -1, 0, 1, 2, 3, 4, 5},
                            ytick={1},                  % 仅在 y=1 处显示一个刻度
                            clip=false,
                        ]
                        % 使用 const plot 绘制分段常数函数,非常适合阶跃信号
                        \addplot[
                            const plot, % 使用阶梯图样式
                            blue,
                            thick,
                        ] coordinates {
                            (-4.5, 0) % 在 t<0 区间,值为0
                            (0, 1)    % 在 t=0 时,跳变到1
                            (5.5, 1)  % 在 t>0 区间,值保持为1
                        };

                        % 为了更精确地表示 t=0 处的跳变点
                        % 在 (0,0) 处画一个空心圆,表示不包含该点
                        \draw[fill=white, blue, thick] (axis cs:0,0) circle (2pt);
                        % 在 (0,1) 处画一个实心圆,表示包含该点
                        \fill[blue] (axis cs:0,1) circle (2pt);
                        \end{axis}
                    }
                    {CT Unit Step: $u(t)$}
                    {fig:continuous_unit_step}
        From this example, you can find that, we use \textcolor{red}{Circle Brakets} to 
        enclose the variable $t$. 
    
    \newpage
    \subsection{Signal Addition}
        Before the official start of our work, let's introduce \textcolor{red}{shifted impulses}.

        Look at the following pictures:
            \inserttikzpicture
                {
                    \pgfplotsset{
                        discrete plot/.style={
                            ycomb,           % 绘制针状图
                            mark=*,          % 数据点标记为实心圆
                            mark size=2pt,   % 标记大小
                            line width=1pt,  % 线条宽度
                            axis lines=middle, % 坐标轴在中间交叉
                            xmin=-3.5, xmax=3.5, % x轴范围
                            ymin=-0.2, ymax=1.0, % y轴范围
                            xtick={-3,-2,-1,0,1,2,3}, % x轴刻度
                            ytick={1},               % y轴刻度
                            enlarge x limits=0.1,    % x轴范围适当放大
                            enlarge y limits=0.2,    % y轴范围适当放大
                            title style={yshift=0.5em}, % 标题位置
                            xlabel={$n$},
                            xlabel style={
                                at={(ticklabel* cs:1.0)},
                                anchor=north west,
                            },
                            ylabel style={
                                at={(ticklabel* cs:1.0)},
                                anchor=east,
                                rotate=0,
                            },
                        }
                    }
                    % 上图:delta[n]
                    \begin{axis}[
                        discrete plot,
                        ylabel={$\delta[n]$},
                        at={(0, 5.5cm)} ]  % 增加垂直位移,为下方图像的箭头留出足够空间
                    \addplot+ coordinates {
                        (-3,0) (-2,0) (-1,0) (0,1) (1,0) (2,0) (3,0)
                    };
                    \end{axis}
                    % 下图:delta[n]
                    \begin{axis}[
                        discrete plot,
                        ylabel={$\delta[n-1]$},
                        %title={时移后的单位脉冲信号 (Time-Shifted Signal)},
                        at={(0, 0cm)}]
                    % 这样只有针状图和数据点是绿色的,坐标轴保持默认的黑色
                    \addplot+[green] coordinates {
                        (-3,0) (-2,0) (-1,0) (0,0) (1,1) (2,0) (3,0)
                    };
                    \end{axis}
                }
                {Shifted Impulses: $\delta[n]$ and $\delta[n-1]$}
                {fig:shifted_impulses}
        When we operate on the time variable $n$ by adding or substracting it,
        the \textcolor{red}{time axis} remains unchanged, 
        only the function graph moves left or right.

        Now suppose we carry out the following operations:
                    \begin{equation}
                        x[n] = \delta[n] + \delta[n-1]
                    \end{equation}
        what will we get? We will get the following Lollipop:
                \inserttikzpicture
                    {
                        % 定义一个通用的离散信号绘图样式
                        \pgfplotsset{
                            discrete plot/.style={
                                ycomb,           % 绘制针状图
                                mark=*,          % 数据点标记为实心圆
                                mark size=2pt,   % 标记大小
                                line width=1.2pt,  % 线条加粗一点
                                axis lines=middle,
                                xmin=-3.5, xmax=3.5,
                                ymin=-0.2, ymax=1.5,
                                xtick={-3,-2,-1,0,1,2,3},
                                ytick={1},
                                enlarge x limits=0.1,
                                enlarge y limits=0.2,
                                xlabel={$n$},
                                xlabel style={at={(ticklabel* cs:1.0)}, anchor=north west},
                                ylabel style={at={(ticklabel* cs:1.0)}, anchor=east, rotate=0},
                            }
                        }
                        \begin{axis}[
                            discrete plot, % 应用上面定义的样式
                            %title={信号叠加 (Signal Addition: $\delta[n] + \delta[n-1]$)},
                            ylabel={$x[n]$}
                        ]
                        % 策略: 分层绘制
                        % 1. 绘制 δ[n] 对应的部分 (蓝色)
                        %    只提供 n=0 处的坐标点,pgfplots 会自动绘制从 y=0 到该点的 "ycomb" 线
                        \addplot+[blue] coordinates {(0,1)};
                        % 2. 绘制 δ[n-1] 对应的部分 (绿色)
                        %    只提供 n=1 处的坐标点
                        \addplot+[green] coordinates {(1,1)};
                        % 3. 绘制其他整数点上的零值标记 (黑色)
                        %    使用 'only marks' 选项,这样就不会画出长度为零的 ycomb 线,只保留标记点
                        \addplot+[black, only marks] coordinates {
                            (-3,0) (-2,0) (-1,0) (2,0) (3,0)
                        };
                        \end{axis}
                    }
                    {Signal Addition: $\delta[n] + \delta[n-1]$}
                    {fig:signal_addition}

        So remember: 
        
        \textbf{An impulse is not a scalar, it's a vector or tensor!}
        
        \textbf{Signal addition means adding up the values of each signal at each time point!}
        
        \subsection{Express $u[n]$ with $\delta[n]$}
        Now do you know how to use $\delta[n]$ to express $u[n]$? 
        
        Yes, we can give:
                        \begin{equation}
                            \begin{aligned}
                                u[n] = \quad & \delta[n] \\
                                    +  & \delta[n-1] \\
                                    +  & \delta[n-2] \\
                                    +  & \delta[n-3] \\
                                    +  & \cdots \\
                                    +  & \delta[n-k] \\
                            \end{aligned}
                        \end{equation}
        each shifted impulses is a vector.
        And we can give a more concise version:
                        \begin{equation}
                            u[n] = \sum_{k=0}^{\infty} \delta[n-k]
                        \end{equation}
        
        Here, it is necessary for us to emphasize the meaning of \textcolor{red}{=}: at every time point, the value of left signal is equal to value of the right signal.

        Now let's take a look at the formula from another perspective:
                        \begin{equation}
                            u[n] = \sum_{k=0}^{\infty} \delta[n-k]
                        \end{equation}
        Make the following variable substitutions:
                        \begin{equation}
                            l = n -k 
                        \end{equation}
        Then:
                        \begin{equation}
                            k = n -l 
                        \end{equation}
        Substitute back, eliminate the variable $k$, change the upper and lower bounds of the summation:
                        \begin{equation}
                            u[n] = \sum_{l=-\infty}^{n} \delta[l]
                        \end{equation}

        How can we understand the equation above? First of all, we should know:
                        \begin{center}
                            \textcolor{red}{When we use the fixed variable $n$, we consider it as a signal, a function or a vector. (e.g. $f(x)$)};

                            \textcolor{red}{When we use another variable or a scalar, we might consider it as a value. (e.g. $f(10)$ or $f(k)$)}
                        \end{center}
        so the above equation is a definition, or represented as a signal.

        So how can we calculate $u[10]$ using the equation? We do as follows:
                        \begin{equation}
                            u[10] = \cdots + \delta[-100] + \cdots + \delta[0] + \cdots + \delta[10]=1
                        \end{equation}
        and we can find, only $\delta[0] = 1$, other components are all 0.

        We can regard the $\sum$ as a big net, only the net's bounds catch the 0
        (the 0 is between $-\infty$ and $n$), then the result is 1:
                        \begin{equation}
                            u[n] = 
                            \left\{
                                \begin{aligned}
                                    &0, \quad n \leq 0 \quad (0 \notin (-\infty, n) )\\
                                    &1, \quad n = 0\\
                                    &1, \quad n > 0\quad (0 \in (-\infty, n) )
                                \end{aligned}
                            \right.
                        \end{equation}
                        
    \subsection{Integral and Derivative in DT}
        Now let's re-examine the formula:
                        \begin{equation}
                            u[n] = \sum_{l=-\infty}^{n} \delta[l]
                        \end{equation}
        we could say: In DT, the unit step is the cumulative sum of $\delta$.
        In other words, it's the DT integral of $\delta$.

        Then, you can easily get: In DT, $\delta$ is the derivative of the unit step.

        How can we verify the above idea? We can try to express $\delta[n]$ with $u[n]$.

        Look at the following pictures:
            \inserttikzpicture
                {
                    \pgfplotsset{
                            discrete plot/.style={
                                ycomb,           % 绘制针状图
                                mark=*,          % 数据点标记为实心圆
                                mark size=2pt,   % 标记大小
                                line width=1pt,  % 线条宽度
                                axis lines=middle, % 坐标轴在中间交叉
                                xmin=-3.5, xmax=5.5, % x轴范围
                                ymin=-0.2, ymax=1.1, % y轴范围稍稍扩大,让'1'的刻度更清晰
                                xtick={-3,-2,-1,0,1,2,3,4,5}, % x轴刻度
                                ytick={1},               % y轴刻度
                                enlarge x limits=0.1,    % x轴范围适当放大
                                enlarge y limits=0.2,    % y轴范围适当放大
                                title style={yshift=0.5em}, % 标题位置
                                xlabel={$n$},
                                xlabel style={
                                    at={(ticklabel* cs:1.0)},
                                    anchor=north west,
                                },
                                ylabel style={
                                    at={(ticklabel* cs:1.0)},
                                    anchor=east,
                                    rotate=0,
                                },
                            }
                        }
                        % ----- 上图: u[n] -----
                        \begin{axis}[
                            discrete plot,
                            ylabel={$u[n]$},
                            %title={单位阶跃信号 (Unit Step Signal)},
                            at={(0, 5.0cm)} % 调整了间距以适应标题
                        ]
                        % 使用蓝色绘制 u[n] 的数据点
                        \addplot+[blue] coordinates {
                            (-3,0) (-2,0) (-1,0) (0,1) (1,1) (2,1) (3,1)(4,1)(5,1)
                        };
                        \end{axis}
                        % ----- 下图: u[n-1] -----
                        \begin{axis}[
                            discrete plot,
                            ylabel={$u[n-1]$},
                            %title={时移阶跃信号 (Time-Shifted Step)},
                            at={(0, 0cm)}
                        ]
                        % 使用绿色绘制 u[n-1] 的数据点
                        \addplot+[green] coordinates {
                            (-3,0) (-2,0) (-1,0) (0,0) (1,1) (2,1) (3,1)(4,1)(5,1)
                        };
                        \end{axis}
                    }
                % 第二、三个大括号,分别是总标题和标签
                {$u[n]$ and $u[n-1]$}
                {fig:shifted_steps}
        So you can easily get:
            \begin{equation}
                \delta[n] = \frac{u[n] - u[n-1]}{1}
            \end{equation}
        If we substitute \textcolor{red}{1} to $\Delta$, we get:
            \begin{equation}
                \delta[n] = \frac{u[n] - u[n-1]}{\Delta}
            \end{equation}
        Actually, it's the definition of the limit when we operate:
            \begin{equation}
                \Delta \rightarrow 0
            \end{equation}
        However, in DT, we cann't get a scalar smaller than 1, because we only use integers.

        To sum up:
            \inserttikzpicture
                {
                    \tikzset{relation_node/.style={
                    circle,          % 形状为圆形
                    draw,            % 绘制边框
                    thick,           % 边框加粗
                    minimum size=1cm % 最小尺寸
                }}
            
                % 2. 放置左右两个节点
                \node[relation_node] (delta) at (0,0) {$\delta[n]$};
                \node[relation_node] (u) at (4,0) {$u[n]$}; 
                % 3. 绘制上方的线(从 delta 指向 u),上方显示“求和(积分)”
                \draw[
                    ->,               % 单向箭头,指向右
                    thick,            % 线条加粗
                    shorten <=2pt,    
                    shorten >=2pt,
                    transform canvas={yshift=2.5pt}     
                ] 
                    (delta) -- (u) % 从delta的右上角到u的左上角,使线条在节点上方
                    node[midway, above=1pt] {$\int$ or $\sum$};
                
                % 4. 绘制下方的线(从 u 指向 delta),下方显示“差分”
                \draw[
                    <-,               % 单向箭头,指向左 (或者用 -> 加反向路径 (u) -- (delta))
                    thick,            % 线条加粗
                    shorten <=2pt,    
                    shorten >=2pt,
                    transform canvas={yshift=-2.5pt}     
                ] 
                (delta) -- (u) % 从delta的右下角到u的左下角,使线条在节点下方
                node[midway, below=1pt] {$\frac{d}{dt}$ or $\nabla$};
                }
                {Relation of $\delta[n]$ and $u[n]$}
                {fig:relation_delta_u}
        \noindent In this picture, the signal can in DT or CT.

        \subsection{Why do We Need Impulses?}
            In the subsections above, we spent a considerable amount of time explaining Impulses.
            So why do we need it?

            The answer is: \textbf{Any signal can be represented as a linear combination of shifted impulses.}

            Let's give an example first:
                \inserttikzpicture
                    {
                            \begin{axis}[
                                % --- 坐标轴样式 (黑色) ---
                                axis lines=middle,              % 坐标轴在原点 (0,0) 相交
                                xlabel={$n$},                   % 横轴名称
                                ylabel={$x[n]$},                % 纵轴名称
                                %title={离散信号火柴杆图 (Stem Plot)}, % 添加一个标题
                                % --- 坐标轴范围与刻度 ---
                                xmin=-2.5, xmax=3.5,            % x 轴范围
                                ymin=-3.5, ymax=3.5,            % y 轴范围
                                xtick={-2, -1, 0, 1, 2, 3},     % x 轴刻度点
                                ytick={-3, -2, -1, 0, 1, 2, 3}, % y 轴刻度点
                                % --- 标签位置样式 (保持美观) ---
                                xlabel style={at={(ticklabel* cs:1.0)}, anchor=north west},
                                ylabel style={at={(ticklabel* cs:1.0)}, anchor=east, rotate=0},
                                % --- 确保坐标轴所有元素为黑色 (pgfplots 默认值) ---
                                axis line style={black},
                                tick label style={black},
                                label style={black},
                                title style={black}
                            ]
                            % --- 绘制数据点 (蓝色) ---
                            \addplot+[
                                ycomb,          % 关键:绘制火柴杆/针状图 (stem plot)
                                blue,           % 将此图的颜色设为蓝色
                                mark=*,         % 在每个数据点末端添加一个实心圆标记
                                mark size=2pt,  % 标记大小
                                line width=1.2pt  % 火柴杆的粗细
                            ] 
                            coordinates {
                                (-1, 2)
                                (0, 0)
                                (1, -2)
                                (2, 3)
                            };
                            \end{axis}
                    }
                    {Example: $x[n]$ with Finite Components}
                    {fig:example_x_finite}
            \noindent Can you write $x[n]$ in terms of shifted $\delta[n]$?

            Yes, the answer is:
                \begin{equation}
                    \begin{aligned}
                        x[n] = & 2\delta[n+1]\\
                              -& 2\delta[n-1]\\
                              +& 3\delta[n-2] 
                    \end{aligned}
                \end{equation}
            
            Then what will happen if $x[n]$ has infinite components? For example:
                \inserttikzpicture
                    {
                        \begin{axis}[
                            % --- 坐标轴样式 ---
                            axis x line=middle, 
                            axis y line=none,
                            xlabel={$n$},
                            % --- 坐标轴范围与刻度 ---
                            % 扩大 xmin 和 xmax 为省略号留出空间
                            xmin=-2.5, xmax=4.5,
                            ymin=0, ymax=3,
                            xtick={-1, 0, 1, 2, 3},
                            ytick=\empty,
                            % --- 标签位置样式 ---
                            xlabel style={at={(ticklabel* cs:1.0)}, anchor=north west},
                            % --- 自定义节点(标签)的样式 ---
                            nodes near coords={\pgfplotspointmeta}, 
                            every node near coord/.style={
                                anchor=east, 
                                rotate=0,
                                black,
                                font=\small
                            }
                        ]
                        % --- 绘制火柴杆和标签 ---
                        \addplot+[
                            ycomb, blue, mark=*, mark size=2pt, line width=1.2pt
                        ] 
                        table[
                            header=false,
                            x index=0,              % 使用第 0 列 (第一列) 作为 x
                            y index=1,              % 使用第 1 列 (第二列) 作为 y
                            meta index=2,             % 使用第 2 列 (第三列) 作为标签元数据
                            point meta=explicit symbolic,
                            row sep=crcr % 修正了行分隔符
                        ] {
                            % n     height   label
                            -1      1          {$x[-1]$} \\
                            0      1.2        {$x[0]$}  \\
                            1      1.5        {$x[1]$}  \\
                            2      2.0        {$x[2]$}  \\
                            3      1.4        {$x[3]$}  \\
                        };
                        % --- 手动添加省略号 ---
                        % 1. 在左侧添加省略号
                        %    'at (axis cs:-2, 0.5)' 将节点放置在坐标 (-2, 0.5) 处
                        %    'axis cs' 表示使用的是坐标轴坐标系 (axis coordinate system)
                        \node at (axis cs:-2, 0.5) {\dots};
                        % 2. 在右侧添加省略号
                        \node at (axis cs:4, 0.5) {\dots};
                        \end{axis}
                    }
                    {Example: $x[n]$ with infinite Components}
                    {fig:example_x_infinite}
            \noindent By the same token, we have:
                \begin{equation}
                    \begin{aligned}
                        x[n] = & \cdots
                              +& x[-1]\delta[n+1]\\
                              +& x[0]\delta[n]\\
                              +& x[1]\delta[n-1]\\
                              +& x[2]\delta[n-2]\\
                              +& x[3]\delta[n-3]\\
                              +& \cdots 
                    \end{aligned}
                \end{equation}
            And we can also give a more concise version:
                \begin{equation}
                    x[n] = \sum_{k=-\infty}^{\infty} x[k]\delta[n-k]
                \end{equation}
            Here, $x[k]$ is a scalar, represented as the value of $x[n]$ at time $k$. 
            Both $x[n]$ and $\delta[n-k]$ are signals.

            Now suppose we want to get $x[10]$, then we have:
                \begin{equation}
                    x[10] = \sum_{k=-\infty}^{\infty} x[k]\delta[10-k]
                \end{equation}
            You can easily find that, only $k=10$ that $\delta[10-k]=1$, otherwise is 0.
            
            So we use the formula to filter the $x[k]$ out, and we can also say: $\delta[n]$ has a screening nature.

    \subsection{Introduction to System}
        \textbf{Definition}: 
        
        A system recives a signal $x$ as an input, process it, and then outputs another signal $y$.
        As a convention we use $H$ to represent the system:
                \inserttikzpicture
                {
                    \tikzset{
                        node distance=2.5cm,
                        block/.style={
                            rectangle, 
                            draw, 
                            thick, 
                            minimum height=1.5cm, 
                            minimum width=1.5cm, 
                            text centered
                        },
                        text_node/.style={
                            align=center
                        },
                        arrow/.style={
                            -Stealth, % 使用 arrows.meta 库定义的箭头样式
                            thick
                        }
                    }
                    % 2. 将原代码中的节点和绘图指令放在这里
                    % 定义各个节点 (Node)
                    \node[text_node] (input) {input signal $x$\\DT or CT};
                    \node[block, right=of input] (system) {H};
                    \node[text_node, right=of system] (output) {output signal $y$\\DT or CT};
                    \node[below=0.5cm of system] (system_label) {\textbf{system}};
                    % 绘制箭头连接节点
                    \draw[arrow] (input) -- (system);
                    \draw[arrow] (system) -- (output);
                    \draw[arrow] (system.south) -- (system_label.north); % 连接锚点以获得更精确的箭头位置
                }
                {System, Input and Output}
                {fig:system_model}

        So what exactly is a system?
        In a word, systems are functions. So a system must have three elements of functions:
                \begin{itemize}
                    \item Domain: $X$, Input spaces, a set of input signals;
                    \item Range: $Y$, Output spaces, a set of output signals;
                    \item Mapping relationship: $H$, a rule operating on a signal.
                \end{itemize}
        Give the relation pictures as follows:
            \inserttikzpicture
                {
                    % 定义一些基本样式
                    \tikzset{
                        space/.style={
                            ellipse, 
                            draw, 
                            thick, 
                            minimum width=2.5cm, 
                            minimum height=7.5cm,
                            label={[font=\Large, anchor=south]north:#1}
                        },
                        element/.style={
                            inner sep=1pt,
                            font=\normalsize
                        },
                        arrow/.style={
                            -Stealth, 
                            thick
                        }
                    }
                    % 1. 绘制输入空间和输出空间
                    \node[space=X] (input_space) {};
                    \node[space=Y, right=4cm of input_space] (output_space) {};
                    % 2. 输入空间内的元素
                    \node[element] at (input_space.north) [yshift=-1cm] (x_dots_top) {$\dots$};
                    \node[element, below=0.7cm of x_dots_top] (x1) {$x_1$};
                    \node[element, below=0.7cm of x1] (x2) {$x_2$};
                    \node[element, below=0.7cm of x2] (x3) {$x_3$};
                    \node[element, below=0.7cm of x3] (x4) {$x_4$};
                    \node[element, below=0.7cm of x4] (x5) {$x_5$};
                    \node[element, below=0.7cm of x5] (x_dots_bottom) {$\dots$};
                    % 3. 输出空间内的元素 (根据 x 元素的位置进行精确对齐)
                    \node[element] at (x1 -| output_space.center) (y1) {$y_1$};
                    \node[element] at (x2 -| output_space.center) (y2) {$y_2$};
                    \node[element] at (x4 -| output_space.center) (y3) {$y_3$};
                    \node[element, above=1cm of y1] (y_dots_top) {$\dots$};
                    \node[element, below=1cm of y3] (y_dots_bottom) {$\dots$};
                    % 4. 绘制曲线箭头 (根据最新建议优化)
                    \draw[arrow] (x1) to[bend left=30] (y1);
                    % 优化点2:让 x2->y2 的箭头(上)和 x3->y3 的箭头(下)分开
                    \draw[arrow] (x2) to[bend left=30] (y2);  % 向上弯曲
                    \draw[arrow] (x3) to[bend right=30] (y3); % 向下弯曲,避免交叉
                    % 优化点1:让指向y3的箭头弯曲方向一致
                    \draw[arrow] (x4) to[bend right=40] (y3); % 向下弯曲
                    \draw[arrow] (x5) to[bend right=50] (y3); % 同样向下弯曲,角度更大以分开
                    % 5. 在顶部中心位置添加唯一的标签, 与 X 和 Y 的基线对齐
                    \node[font=\large] at ($(input_space.north)!0.5!(output_space.north)$) {Mapping rules};
                }
                {Input Space, Output Space and Mapping Rules}
                {fig:mapping_relation}

                \textbf{Note}: Multiple $x$s are allowed to correspond to one $y$, 
                one $x$ is not allowed to correspond to multiple $y$s.

                In most of the cases, we pay attention to: single input, single output.

                
    
        
        
        
        
        
        
        
            



        


        
        

        
        
                        
        
        
        








        
        

        
        

        


        