\section{Lecture1}
    \subsection{What is a Signal}
        In this subsection, we will introduce some notions in \textbf{SS} course.

        First, what is a signal? A signal is a \textbf{function}. Suppose \textbf{x} is a signal,
        then it can represent the following mapping relationship:
            \begin{equation}
                \begin{aligned}
                    \mathbb{R}(\mathrm{reals}) &\longrightarrow \mathbb{R}(\mathrm{reals}) \\
                    \mathbb{Z}(\mathrm{integers}) &\longrightarrow \mathbb{R}(\mathrm{reals}) \\
                    \mathbb{Z}(\mathrm{reals}) &\longrightarrow \mathbb{C}(\mathrm{complexes})
                \end{aligned}
            \end{equation}
        In the above formula,  
        we call the left \textbf{domain or input space}, we call the right \textbf{range or output sapce},
        referring to the definition in the function,

        At the same time, an element in domain is called \textbf{independent variable}, an element in range is called \textbf{dependent variable or the value of function}.
        
        After we give the doamin and the range, we need a \textbf{rule} to map every element in the domain to the range,
        which says what we operate on the the independent variable. The rule is also called the \textbf{function relation}.
        
        For example, considering the following euqation:
            \begin{equation}
                x(n) = \cos \frac{\pi}{4} n
            \end{equation}
        we have the following relations:
            \begin{itemize}
                \item \textbf{domain}: $\mathbb{Z}$, a set of all integers;
                \item \textbf{independent variable}: $n$, integers;
                \item \textbf{range}: all real numbers in [-1,1];
                \item \textbf{value of function}: $x(n)$;
                \item \textbf{function relation}: \textbf{\emph{x}}, $  \cos \frac{\pi}{4}[\cdot] $
            \end{itemize}
    
    \newpage
    \subsection{DT and CT}
        Now let's introduce \textbf{DT} and \textbf{CT}.

        For a signal \textbf{x}, if the domain is $\mathbb{Z}$ or the set of all integers, 
        then the signal \textbf{x} is called \textbf{Discrete-Time} signal (\textbf{DT}).
        For example:
            \begin{equation}
                x(n) = \cos \frac{\pi}{4} n,\quad n \in \mathbb{Z}
            \end{equation}
        
        For a signal \textbf{\emph{x}}, if the domain is $\mathbb{R}$ or the set of all reals, 
        then the signal \textbf{\emph{x}} is called \textbf{Continuous-Time} signal (\textbf{CT}).
        Also, you can call \textbf{CT signal} as \textbf{Analog signal}.
        For example:
            \begin{equation}
                x(t) = \mathrm{e}^{-t},\quad t \in \mathbb{R}
            \end{equation}

        Meanwhile:
            \begin{itemize}
                \item if we use $n$ as the independent variable, that means the signal is \textbf{DT};
                \item if we use $t$ as the independent variable, that means the signal is \textbf{CT};
            \end{itemize}
        these are the conventions.

        \textbf{Supplement}: Does the signal have to be the function of \textbf{only time}? The answer is \textbf{No}.
        For example, the independent variables can be \textbf{space coordinates}.
    
    \subsection{Discrete-Time Signal and Digital Signal}
        Sometimes, we don't distinguish \textbf{digital signal} and \textbf{discrete-time signal}, 
        but they still have obvious differences.

        In terms of digital signal, it still meets the definition of function. 
        However, everything (including independent variable in domain and dependent variable in range) are represented by digits.

        In terms of discrete-time signal, the independent variable in domain is discrete, 
        but the values of the function can be reals (such as infinite recurring decimal).

        But for the computers, we can't have infinite precision, so we need to quantify the numbers, 
        which will cause us to lose accurcy.

    \newpage
    \subsection{Examples and Conventions}
        Let's give somen examples about \textbf{DT} and \textbf{CT}.

        First, there is a continuous time example, a speech signal:
            \inserttikzpicture
                { % 第一个参数: TikZ 绘图代码
                    \begin{axis}[
                        compat=1.17,
                        axis lines=middle,
                        xtick=\empty,
                        ytick=\empty,
                        ymin=-2,
                        ymax=2.2, 
                        xmin=0,
                        xmax=10.5,
                        clip=false, 
                        xlabel=$t$,
                        xlabel style={
                            at={(ticklabel* cs:1.0)},
                            anchor=north,
                        },
                        ylabel={$x(t)$},
                        ylabel style={
                            at={(ticklabel* cs:1.0)},
                            anchor=east,
                            rotate=0,
                        },
                    ]
                        \addplot[
                            domain=0:10,
                            samples=200,
                            smooth,
                            thick,
                            blue,
                        ] {sin(deg(2*x)) * (1 + 0.5*sin(deg(5*x)))};
                    \end{axis}
                }
                {A Speech Signal}
                {fig:speech_signal}
        
            \inserttikzpicture
                {
                    % 使用您提供的坐标轴样式和布局
                    \begin{axis}[
                        compat=1.17,
                        axis lines=middle,
                        xtick=\empty,  % 移除 x 轴刻度
                        ytick=\empty,  % 移除 y 轴刻度
                        % 调整坐标轴范围以适应 e^{-t} 并在 t<0 时有足够空间
                        ymin=-1,
                        ymax=8, 
                        xmin=-2.5,
                        xmax=5,
                        clip=false, 
                        xlabel=$t$,
                        xlabel style={
                            at={(ticklabel* cs:1.0)},
                            anchor=north,
                        },
                        ylabel={$x(t)$},
                        ylabel style={
                            at={(ticklabel* cs:1.0)},
                            anchor=east,
                            rotate=0,
                        },
                    ]
                    % 绘制函数 x(t) = e^{-t},并扩展定义域至负数
                    \addplot[
                        domain=-2:4.5, % 设置绘图的定义域,包含负数
                        samples=200,
                        smooth,
                        thick,
                        blue,
                    ] {exp(-x)};
                    \end{axis}
                } 
                {$x(t)=e^{-t}$}
                {fig:exp_decay}
        \noindent For the \textbf{CT} signal, we have the following conventions:
            \begin{itemize}
                \item $x$ at time $t$: $x(t)$, refers to a specific value of the signal at time t;
                \item $x$ without any arguments: \textbf{\emph{x}}, refers to the entire signal in $(-\infty, \infty)$;
            \end{itemize}
        However, in practice, when we define a signal, $x(t)$ also refers to the entire signal, not a specific value.

        Second, there is a discrete time signal:
            \inserttikzpicture
                {
                    % 使用 pgfplots 环境绘制 stem plot
                    \begin{axis}[
                        compat=1.17,
                        axis lines=middle,          % 坐标轴在原点交汇
                        ylabel={$x(n)$},            % y轴标签
                        ylabel style={              % y轴标签样式,与您之前的示例保持一致
                            at={(ticklabel* cs:1.0)},
                            anchor=east,
                            rotate=0,
                        },
                        xlabel=$n$,                 % x轴标签 (离散时间索引)
                        xlabel style={              % x轴标签样式
                            at={(ticklabel* cs:1.0)},
                            anchor=north west,      % 微调位置防止与坐标轴箭头重叠
                        },
                        % 为离散时间信号设置坐标轴范围
                        ymin=-2, ymax=3,
                        xmin=-3.5, xmax=6.5,
                        % 在x轴上显示整数刻度,这对于离散信号至关重要
                        xtick={-3, -2, -1, 0, 1, 2, 3, 4, 5, 6},
                        ytick=\empty,               % 保持y轴的简洁,移除刻度
                        clip=false,                 % 允许绘图内容超出坐标轴范围
                    ]
                    % 使用 ycomb 样式来绘制“火柴杆”
                    % mark=* 表示在每个数据点顶端添加一个实心圆点
                    \addplot+[
                        ycomb,
                        blue, % 设置颜色
                        thick, % 加粗线条
                        mark=*,
                        mark options={fill=blue}, % 将标记点用蓝色填充
                    ] coordinates {
                        (-2, 1)
                        (-1, 2)
                        (0, 1.5)
                        (1, -1)
                        (2, -1.5)
                        (3, 1)
                        (4, 0)
                        (5, 0.5)
                    };
                    \end{axis}
                }
                {An Example for \textbf{DT} signal}
                {fig:discrete_signal_example}
        \noindent The picture above is called \textbf{lollipops} or \textbf{stem plots}, 
        which is designed specifically for the \textbf{DT} signal. 
        We can find that, it's discrete on the domain.
        
        

        


        